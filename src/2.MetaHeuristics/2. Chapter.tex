\documentclass[../main.tex]{subfiles}

\begin{document}

\chapter{Мета-эвристики}\label{metaheuristicchapter}

В данной главе мы рассмотрим первый подход к решению задачи, часто называемый <<эволюционным>>. В этом подходе мы никак не будем использовать формализацию процесса принятия решений (а следовательно, не будем использовать какие-либо результаты, связанные с изучением MDP) и будем относиться к задаче как к black-box оптимизации: мы можем отправить в среду поиграть какую-то стратегию и узнать, сколько примерно она набирает, и задача алгоритма оптимизации состоит в том, чтобы на основе лишь этой информации предлагать, какие стратегии следует попробовать следующими. 

\import{2.MetaHeuristics/}{2.1.Baselines.tex}
\import{2.MetaHeuristics/}{2.2.EvolutionStrategies.tex}

\end{document}